مراد از واژهٔ «اسلام» دین پیامبر خاتم حضرت محمد{\صلی}و پیروان ایشان یعنی «مسلمانان» است. همان طور که پیداست در هر دو واژهٔ‌ اسلام و مسلمان، سه حرف سین، لام و میم وجود دارد. در زبان عربی بیشتر واژه‌ها ریشه‌ای سه‌حرفی دارند. از همین رو، این دو واژه مشتق از یک ریشهٔ واژگانی هستند. البته س-ل-م ریشهٔ واژهٔ دیگری نیز است که آن واژه یکی از نام‌های خداوند، یعنی «السلام»،  آرام و صلح‌طلب، است؛ کما این که در یکی از دعاهای منسوب به پیامبر اکرم{\صلی}\LTRfootnote{
	\lr{Islahi, Mohammad Yusuf. ``Etiquettes of Life in Islam.’’ New edited English version of Urdu A’dab-E-Zindagi. Markazi Maktaba Islam. page 106, 1997.}} آمده است:
\begin{itemize}
	\item[]
	{\قرآن
		اللَّهُمَّ أَنْتَ السَّلَامُ وَمِنْكَ السَّلَامُ تَبَارَكْتَ ذَا الْجَلَالِ وَالْإِكْرَام}
	
	{\ترجمه پروردگارا تو خود آرامش هستی و آرامش از جانب توست و تو بابرکت هستی ای صاحب شکوه و عظمت.}
\end{itemize}

اسلام که در معنای واژگانی به معنای سرسپردگی است، بنا را بر حصول آرامش در سایهٔ سرسپردگیِ حقیقت «السلام» گذاشته است: سرسپردگی در برابر آرامش مطلق موجب آرامش می‌شود. تنها یک راه برای رسیدن به خداوند وجود دارد و همان طور که پیش‌تر گفتیم، ما ناگزیر از رفتنیم؛ چه بخواهیم و چه نخواهیم ما به سمت او می‌رویم. مسلمان بودن معنایش رفتن در این مسیر با آغوش باز است و دگرگونی‌ای که پس از این سرسپردگی اتفاق می‌افتد، در زندگی همین دنیا قابل مشاهده است. به همین خاطر، همهٔ انسان‌ها این فرصت را دارند که به سرسپردگی خودخواسته در برابر آفریدگارشان برسند. مثلاً در آیهٔ شصت و هفتم سورهٔ آل‌عمران در مورد حضرت ابراهیم{\علیه} آمده است:
\begin{itemize}
	\item[]
	{\قرآن
		مَا كَانَ إِبْرَاهِيمُ يَهُودِيًّا وَلَا نَصْرَانِيًّا وَلَـٰكِن كَانَ حَنِيفًا مُّسْلِمًا وَمَا كَانَ مِنَ الْمُشْرِكِينَ}
	
	{\ترجمه ابراهیم نه یهودی بود و نه نصرانی، بلکه موحّدی خالص و مسلمان بود و هرگز از مشرکان نبود.}
\end{itemize}

قرآن بر این نکته تأکید دارد که باور به پروردگار و جهان پس از مرگ در مرکزیت ادیانِ پیش از اسلام نیز بوده است. در یکی از داستان‌های طولانی در مورد حضرت موسی{\علیه}، در آیهٔ شصت و دوم سورهٔ بقره، چنین توضیحی در میانهٔ روایت داستان آمده است:


\begin{itemize}
	\item[]
	{\قرآن
		إِنَّ الَّذِينَ آمَنُوا وَالَّذِينَ هَادُوا وَالنَّصَارَىٰ وَالصَّابِئِينَ مَنْ آمَنَ بِاللَّـهِ وَالْيَوْمِ الْآخِرِ وَعَمِلَ صَالِحًا فَلَهُمْ أَجْرُهُمْ عِندَ رَبِّهِمْ وَلَا خَوْفٌ عَلَيْهِمْ وَلَا هُمْ يَحْزَنُونَ}
	
	{\ترجمه
		کسانی که ایمان آورده‌اند، و کسانی که به آئین یهود گرویدند و نصاری و صابئان [پیروان یحیی‌] هر گاه به خدا و روز رستاخیز ایمان آورند، و عمل صالح انجام دهند، پاداششان نزد پروردگارشان مسلم است و هیچ‌گونه ترس و اندوهی برای ‌آن‌ها نیست.}
\end{itemize}


آن طور که علامه طباطبایی در جلد اول تفسیر المیزان گفته است، آیهٔ یادشده قاعده‌ای جهان‌شمول را بنا می‌کند. بدین‌صورت  سرسپردگی در مقابل ذات اقدس الهی برای همهٔ انسان‌ها امکان‌پذیر می‌شود. این اصل جهان‌شمول آن است که به خداوند باور داشته باشیم، برای سفر پس از مرگ آماده شویم و درست عمل کنیم.

از این منظر، اسلامِ امروز فرقی با اسلامی که قبلاً وجود داشته ندارد. وقتی حضرت ابراهیم{\صلی}خانه‌اش را به خاطر دستور پروردگار ترک گفت یا وقتی که به گونه‌ای که ما از آن آگاه نیستیم خداوند را  نیایش می‌کرد، یا وقتی که به مهمانانش ارج می‌نهاد، در حال انجام کار درست بود. همهٔ این نمونه‌هایی که از زندگی ایشان در انجیل و قرآن آمده است به ما راه درست مسلمان بودن را نشان می‌دهد.

پیامبر خاتم{\صلی}به پیروانش گفت که آن‌ها پیرو سنت‌هایی هستند که پیامبران قبلی بنا نهاده‌اند. مسلمانان امروز نیز کارهایی مشابه انجام می‌دهند گرچه در جزئیات متفاوتند. مثلاً ما خانه‌هایمان را برای سفر به شهر مقدس مکه ترک می‌کنیم تا طبق فرمان خداوند به حج برویم. طبق دستور پیامبر{\صلی} به صورت منظم نماز می‌خوانیم.  سعی می‌کنیم که مهمانمان را گرامی بداریم، همان طور که حضرت ابراهیم{\علیه}این گونه عمل می‌کرد. این مسائل با جزئیات در قرآن و تعلیمات منسوب به پیامبر اکرم{\صلی}آمده است. نکتهٔ مهم اما آن است که هیچ‌کدام از این‌ها چیز جدیدی نیست. اسلام همیشه واکنش طبیعی انسان در مواجهه با پروردگار بوده، هست و خواهد بود. در سوره‌ٔ عصر، که یکی از کوتاه‌ترین سوره‌های قرآن است، این مسأله بیان شده است. الشافعی، یکی از امامان  اهل سنت، این سوره را برای راهنمایی بشریت کافی می‌دانست:

\begin{itemize}
	\item[]
	{\قرآن
		
		وَالْعَصْرِ ﴿١﴾ إِنَّ الْإِنسَانَ لَفِي خُسْرٍ ﴿٢﴾ إِلَّا الَّذِينَ آمَنُوا وَعَمِلُوا الصَّالِحَاتِ وَتَوَاصَوْا بِالْحَقِّ وَتَوَاصَوْا بِالصَّبْرِ ﴿٣﴾}
	
	{\ترجمه
		به عصر سوگند (۱) که انسان‌ها همه در زیانند (۲) مگر کسانی که ایمان آورده و اعمال صالح انجام داده‌اند، و یکدیگر را به حق سفارش کرده و یکدیگر را به شکیبایی و استقامت توصیه نموده‌اند. (۳)}
\end{itemize}

همهٔ حرف من در این کتاب همین چیزی است که در آیات سورهٔ عصر آمده است. این که خودم و شما را تشویق به جستجوی حقیقت و پایمردی در این راه کنم.  

روزه گرفتن یکی از مهم‌ترین کارهای تأثیرگذار در زندگی معنوی انسان‌هاست، اما نوع روزه گرفتنی که در قرآن توصیه شده است، بسیار طاقت‌فرساست. این که برای کل ماه رمضان از انتهای سحر تا غروب کامل آفتاب از خوردن، آشامیدن و آمیزش جنسی پرهیز کنیم، کار ساده‌ای نیست. حدود بیست سال از اسلام آوردنم می‌گذرد و هنوز هم که هنوز است روزه گرفتن برایم دشوار است. ولی این کار وقتی که در کنار بقیهٔ مسلمانان انجام شود آسان‌تر ‌‌می‌شود. به اطراف جهان نگاه ‌می‌کنم و ‌‌می‌بینم صدها میلیون مسلمان با هر پیشینه و فرهنگی این کار را انجام می‌دهند. درمی‌یابم که  با وجود دشواری روزه گرفتن،  این کار کاری نیست که شدنی نباشد. اگر یک نوجوان پانزده ساله در مراکش یا پیرزن شصت ساله‌ای در ایران می‌تواند این کار را انجام دهد، پس من هم می‌توانم. روزه گرفتن محکی است برای معنویت انسان، زیرا که نشان می‌دهد  ما چقدر توان  دل کندن موقتی  ازخواسته‌های ابتدایی‌مان به خاطر رسیدن به معنویت داریم. وقتی که بسیاری از افراد با روزه گرفتن‌شان نشان می‌دهند که این کار شدنی است، آن‌هایی که روزه نمی‌گیرند دیگر عذری پذیرفتنی ندارند. تنها راه آن است که به اندازهٔ کافی خواهانِ معنویت باشیم.


اما سؤال اینجاست که چرا باید کسی از خوردن، آشامیدن و آمیزش جنسی چشم‌پوشی کند؟ آیا این‌ها بهترین نعمت‌های زندگی نیستند؟ بله، هستند و دقیقاً به همین دلیل است که چشم‌پوشی موقتی از آن‌ها معنادار می‌شود. ما با چشم‌پوشی از آن‌ها از کسی که آن چیزها را به ما داده قدردانی می‌کنیم. مثلاً فرض کنید کسی که دوست‌تان دارد به شما صد هزار دلار بدهد. او این پول را به شما داده چون خیرِ شما را می‌خواسته است. حالا فرض کنید او روزی بیاید پیش شما و بخواهد همان پولی را که به شما داده از شما قرض بگیرد و قول بدهد که موقع پس دادن پول، مبلغی اضافه‌تر به شما پرداخت کند. هر کسی جز خودخواه‌ترین و بی‌اعتمادترین آدم‌ها حتماً این پیشنهاد را قبول می‌کند و به قول قرض‌گیرنده اعتماد می‌کند. خب، این دقیقاً همان کاری است که خداوند در قرآن، آیهٔ دویست و چهل و پنجم سورهٔ بقره، به آن اشاره دارد:

\begin{itemize}
	\item[]
	{\قرآن
		مَّن ذَا الَّذِي يُقْرِضُ اللَّـهَ قَرْضًا حَسَنًا فَيُضَاعِفَهُ لَهُ أَضْعَافًا كَثِيرَةً  وَاللَّـهُ يَقْبِضُ وَيَبْسُطُ وَإِلَيْهِ تُرْجَعُونَ}
	
	{\ترجمه
		کیست که به خدا «قرض‌الحسنه‌ای» دهد،  تا آن را برای او، چندین‌برابر کند؟ و خداوند است که محدود یا گسترده می‌سازد. و به سوی او باز می‌گردید.}
\end{itemize}

علاوه بر روزه گرفتن در ماه مبارک رمضان، یکی از جنبه‌های اصلی اسلام کار خیریه است. همهٔ‌ مؤمنان در طول تاریخ تشویق به کار خیر شده‌اند و برای پیروان حضرت محمد{\صلی}نیز جز این نبوده است. روزی‌دهنده، «الرزاق»، یکی از نام‌های خداوند است و  او سرچشمهٔ ثروت بی‌پایان است. اگر شما در خانواده‌ای ثروتمند به دنیا آمده باشید، این خداوند بوده که چنین تقدیری را برای شما رقم زده است. اگر ثروت‌تان را با کار سخت به دست آورده باشید، این خداوند بوده که به شما هوش و سلامتی لازم برای کار سخت را داده است. هر چیزی که ظاهراً در تصاحب افراد است در نهایت به آن‌ها تعلق ندارد. در نهایت همه چیز از آن خداست. به همین خاطر خداوند از ما می‌خواهد  بخشی از آن ثروت‌ها را بدهیم تا نشان دهیم اعتقاد ما به او و زندگی پس از مرگ از سر صدق است. در عوض، پروردگار قول داده است که بیشتر از آن چیزی را که داده‌ایم به ما پس بدهد. خداوند در آیهٔ هفتم سورهٔ ابراهیم می‌فرماید {\قرآن«لَئِن شَكَرْتُمْ لَأَزِيدَنَّكُمْ».}\footnote{اگر شکرگزاری کنید، بر شما خواهم افزود.} ما مقداری پول به خداوند به عنوان قرض  می‌دهیم و به قول او اعتماد داریم که او در عوض مقدار بیشتری را به ما پس خواهد داد. در آیهٔ هجدهم سورهٔ حدید آمده است:

\begin{itemize}
	\item[]
	{\قرآن
		إِنَّ الْمُصَّدِّقِينَ وَالْمُصَّدِّقَاتِ وَأَقْرَضُوا اللَّـهَ قَرْضًا حَسَنًا يُضَاعَفُ لَهُمْ وَلَهُمْ أَجْرٌ كَرِيمٌ}
	
	{\ترجمه
		مردان و زنان انفاق‌کننده، و آن‌ها که به خدا «قرض‌الحسنه» دهند، برای آنان مضاعف می‌شود و پاداش پرارزشی دارند.}
\end{itemize}

در قرآن از کار خیر معمولاً به عنوان «زکات» یاد شده است. البته زکات فراتر از مفهوم خیریه است و در واقع قاعده‌ای است که به خیلی از جنبه‌های زندگی مرتبط می‌شود. واژهٔ زکات از نظر ریشه‌شناسی با مفهوم پاکی و خلوص در ارتباط است و در واقع، زکات اشاره به سازوکاری دارد که از طریق آن می‌توانیم همهٔ‌ جنبه‌های زندگی‌مان را پاک و خالص کنیم.  در حدیثی از امام جعفر صادق{\علیه}آمده است:

\begin{itemize}
	\item[]
	
	{\ترجمه
		«خدواند در همهٔ اعضای بدنت و در رشد هر تار مویت و در هر لحظه‌ای زکاتی قرار داده است. زکات چشم آن است که برای آموختن ببینی و نگاهت را از آرزوها و تمایلات بپوشانی. زکات گوش‌ها آن است که به دانش، خِرَد، قرآن، مباحث دینی، مانند ابشارها و انذارها، و هر چیزی که نجات تو در آن است، گوش فرادهی و از شنیدن هر چیزی که نجات تو در آن نیست مانند دروغ و غیبت و مانند آن بپرهیزی. زکات زبان آن است که مشورت مشفقانه به مؤمنان بدهی، غافلان را بیدار کنی، و همیشه مشغول به ذکر و حمد خداوند و مانند آن باشی. زکات دست آن است که نسبت به هر آنچه که خداوند به تو داده است سخاوت‌مند باشی؛ دانش و هر چیزی را که برای مسلمانان در راه اطاعت پروردگارِ بلندمرتبه مفید است بنویسی و دستت را از کارهای نادرست بازداری. زکات پاهایت آن است که آن‌ها را برای ادای حقوق الهی به کار بری، به دیدار مردمان صالح بروی و به جمع‌هایی بروی که یاد خداوند در آن است. قدم در آشتی دادن مردم، صلهٔ رحم، جهاد، و به سمت هر آنچه که  دل و دینت در آن است بروی.»}\footnote{از کتاب «الحقائق فی محاسن الأخلاق» نوشتهٔ ملامحسن فیض کاشانی به نقل از بحارالانوار.}
\end{itemize}

محاسبهٔ زکاتِ ثروتْ کار ساده‌ای است اما محاسبهٔ زکات بدن چالش‌برانگیز است. حداقل کاری که می‌شود کرد آن است که همهٔ همت خود را صرف آن کنیم که عباداتی را که پیامبر اکرم{\صلی}به ما آموخته به درستی انجام دهیم. راه‌های مختلفی برای عبادت در اسلام وجود دارد ولی آن چیزی که در عربی از آن به عنوان «صلاة» (نماز ) یاد می‌شود برای همهٔ مسلمان اجباری است. وقتی می‌گویند مسلمانان پنج مرتبه در روز نماز می‌خوانند به این معناست که مسلمانان «حداقل» پنج بار در روز نماز می‌خوانند. در عمل ممکن است خیلی بیشتر به روش‌های مختلف به عبادت خداوند بپردازند. اما نماز صورت بنیادین نیایش است.

برای ادای نماز باید پوشیده و تمیز باشیم و شعائر آیینی خاصی را هم‌زمان با خواندن نیایش‌هایی خاص به جا بیاوریم. خواندن اولین سوره از قرآن، سورهٔ حمد، بخش مرکزی نماز است. 

\begin{itemize}
	\item[]
	{\قرآن
		بِسْمِ اللَّـهِ الرَّحْمَـٰنِ الرَّحِيمِ ﴿١﴾ الْحَمْدُ لِلَّـهِ رَبِّ الْعَالَمِينَ ﴿٢﴾ الرَّحْمَـٰنِ الرَّحِيمِ ﴿٣﴾ مَالِكِ يَوْمِ الدِّينِ ﴿٤﴾ إِيَّاكَ نَعْبُدُ وَإِيَّاكَ نَسْتَعِينُ ﴿٥﴾ اهْدِنَا الصِّرَاطَ الْمُسْتَقِيمَ ﴿٦﴾ صِرَاطَ الَّذِينَ أَنْعَمْتَ عَلَيْهِمْ غَيْرِ الْمَغْضُوبِ عَلَيْهِمْ وَلَا الضَّالِّينَ ﴿٧﴾}
	
	{\ترجمه
		به نام خداوند بخشندهٔ بخشایشگر (۱) ستایش مخصوص خداوندی است که پروردگار جهانیان است. (۲) بخشنده و بخشایشگر است. (۳)  مالک روز جزاست. (۴) تنها تو را می‌پرستیم و تنها از تو یاری می‌جوییم. (۵) ما را به راه راست هدایت کن. (۶) راه کسانی که آنان را مشمول نعمت خود ساختی، نه کسانی که بر آنان غضب کرده‌ای، و نه گمراهان. (۷)
	}
\end{itemize}

علاوه بر زکات بدن، زکاتِ روز نیز وجود دارد. زمان لازم برای آماده شدن و ادای پنج مرتبه نماز تقریباً بیست و پنج دقیقه در روز است. یعنی هر روز حداقل بیست و پنج دقیقه به خداوند اختصاص دهیم. اگر کسی هر شبانه‌روز شش ساعت بخوابد، با توجه به زمانی که به نماز اختصاص داده است، هنوز هفده ساعت و نیم از وقتش برای کارهای دیگر باقی می‌ماند. آن‌هایی که بیشتر با خداوند در ارتباطند معمولاً زمان بسیار بیشتری از بیست و پنج دقیقه را به نماز اختصاص می‌دهند که البته این کار واجب نیست. پنج وعده نماز همچون روزه گرفتن محک خوبی برای مسلمانی ماست. نماز خواندن زکات روز است، روزه گرفتن زکات سال، و سفر حج زکات عمر است. این مفهومی است که اسلام برای ما ساخته است. این که از آن همه نعمتی که خداوند به ما عطا کرده است، تنها بخش کوچکی‌اش را به او بازگردانیم. هر چه بیشتر قدردان باشیم، بیشتر به خداوند بازمی‌گردانیم. اگر به مفهوم دین که در فصل اول اشاره شد بازگردیم، درمی‌یابیم اسلام راهی است که ما دِین خود را، که همان وجود ما در دنیاست، پس بدهیم. 

شایان گفتن است که نماز، روزه، زیارت و کار خیر همه به زندگی اجتماعی ما در ابعادی بزرگ‌تر مرتبط هستند. نمازهای مسلمانان معمولاً با بقیه به صورت جماعت خوانده می‌شود. در ماه رمضان، بسیاری از مسلمانان همدیگر را به شام دعوت می‌کنند. ما معمولاً کمک‌های خیریه را به کسانی که در نزدیکی ما هستند می‌دهیم و این گونه از نیازهای جامعهٔ کوچک اطراف‌مان باخبر می‌شویم. زیارت نیز هم‌دوش بقیهٔ مسلمانان دیگر، که از جاهای مختلف جهان به حج می‌آیند، صورت می‌گیرد. به همین خاطر، اخلاق به معنای عامش نیز در مرکزیت قاعدهٔ انجام دادن کار درست قرار می‌گیرد. چون ما در جامعه زندگی می‌کنیم، باید به جامعه احترام بگذاریم. مسلمانان نباید مرتکب قتل شوند، دزدی کنند، دروغ بگویند، یا به بقیه تهمت بزنند. خداوند به مسلمانان دستور داده است که به نیازمندان کمک کنند و همیشه در تکاپوی بهتر شدن باشند. در آیات هشتم تا هفدهم سورهٔ بلد آمده است:

\begin{itemize}
	\item[]
	{\قرآن
		أَلَمْ نَجْعَل لَّهُ عَيْنَيْنِ ﴿٨﴾ وَلِسَانًا وَشَفَتَيْنِ ﴿٩﴾ وَهَدَيْنَاهُ النَّجْدَيْنِ ﴿١٠﴾ فَلَا اقْتَحَمَ الْعَقَبَةَ ﴿١١﴾ وَمَا أَدْرَاكَ مَا الْعَقَبَةُ ﴿١٢﴾ فَكُّ رَقَبَةٍ ﴿١٣﴾ أَوْ إِطْعَامٌ فِي يَوْمٍ ذِي مَسْغَبَةٍ ﴿١٤﴾ يَتِيمًا ذَا مَقْرَبَةٍ ﴿١٥﴾ أَوْ مِسْكِينًا ذَا مَتْرَبَةٍ ﴿١٦﴾ ثُمَّ كَانَ مِنَ الَّذِينَ آمَنُوا وَتَوَاصَوْا بِالصَّبْرِ وَتَوَاصَوْا بِالْمَرْحَمَةِ ﴿١٧﴾}
	
	{\ترجمه
		آیا برای او دو چشم قرار ندادیم (۸) و یک زبان و دو لب؟ (۹) و او را به راه خیر و شرّ هدایت کردیم. (۱۰) ولی او از آن گردنهٔ مهم نگذشت. (۱۱) و تو نمی‌دانی آن گردنه چیست. (۱۲) آزاد کردن برده‌ای (۱۳) یا غذا دادن در روز گرسنه‌‌ای (۱۴) یتیمی از خویشاوندان (۱۵) یا مستمندی خاک‌نشین را (۱۶) سپس از کسانی باشد که ایمان آورده و یکدیگر را به شکیبایی و رحمت توصیه می‌کنند. (۱۷)
	}
\end{itemize}

در بخشی از آیهٔ نهم سورهٔ الحشر می‌فرماید {\قرآن «وَيُؤْثِرُونَ عَلَىٰ أَنفُسِهِمْ وَلَوْ كَانَ بِهِمْ خَصَاصَةٌ».}\footnote{در دل خود نیازی به آنچه به مهاجران داده شده احساس نمی‌کنند و آن‌ها را بر خود مقدّم می‌دارند هر چند خودشان بسیار نیازمند باشند.} این کار ساده‌ای نیست که به گرسنه‌ای غذا بدهیم حال آن که خودمان گرسنه هستیم، یا آنقدر نگران وضع مالی دیگران باشیم که خودمان به فقر بیفتیم. این یک آرمان رفیع اخلاقی است.

در زندگی گاهی بسیار به جزئیات دقت می‌کنیم و گاهی بسیار سهل‌انگار می‌شویم. اسلام آمده تا در زندگی‌مان به تعادل برسیم و چیزهای مهم را در اولویت قرار بدهیم. اگر من نماز واجبم را نخوانم، نماز مستحبی شب هیچ فایده‌ای برایم نخواهد داشت. اگر من در تمام سال هیچ کار خیری انجام نداده باشم، احتمالاً نباید نیمی از پولم را در یک روز بدهم چون آن روز حس خوبی نسبت به کار خیر داشته‌ام. معمولاً تغییر باید به تدریج صورت بگیرد و باید قبل از آن که به ثمر بنشیند، خوب ریشه بگیرد. سال‌ها تلاش و صبر برای من لازم بود تا در خواندن نماز تسلط پیدا کنم. باید یاد می‌گرفتم چگونه قرآن را به عربی بخوانم. باید انعطاف بیشتری به خرج می‌دادم تا طریقهٔ‌ درست نشستن موقع نماز را یاد بگیرم. باید بسیاری از عبارات عربی را حفظ می‌کردم و معنایشان را می‌آموختم. با آن که این کارها موقعی برایم دشوار می‌نمود، اکنون خیلی ساده و طبیعی این کارها را انجام می‌دهم. به همین منوال، ازخودگذشتگی به صورت طبیعی در من شکل نمی‌گیرد. زیرا من آن گونه تربیت شدم که به مهارت‌ها و خواسته‌هایم توجه داشته باشم. اسلام به من آموخت که معمولاً کار درست ربط زیادی به آنچه می‌خواهم ندارد، بلکه کار درست آن چیزی است که برای دیگران بهترین کار است. گذشتن از آن چیزی که نفس می‌خواهد اصلاً کار ساده‌ای نیست. اگر من از داشتن ثروت لذت می‌برم، خیلی سخت است که آن ثروت را به کار خیر اختصاص دهم. اگر از وقت آزادم لذت می‌برم، خیلی سخت است که آن را به خواندن نماز اختصاص بدهم. اگر دوست داشته باشم فیلم سینمایی تماشا کنم، اما همسرم از من می‌خواهد قبل از رسیدن مهمان‌ها به خرید بروم، آن‌گاه است که دشواری عمل به کار درست برایم آشکار می‌شود. گذشتن از خودخواهی و رسیدن به ازخودگذشتگی بدون تغییر دادن خود شدنی نیست. این تغییرات پنج مرحلهٔ اساسی دارد که در فصل بعدی بدان‌ها خواهم پرداخت.


