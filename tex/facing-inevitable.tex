مطالب غیرمنتظره در قرآن فراوان دیده می‌شود. یکی از نمونه‌های آن اشاره به واژهٔ مرگ پیش از واژهٔ زندگی است.
\begin{itemize}
	\item[]
	{\قرآن
		تَبَارَكَ الَّذِي بِيَدِهِ الْمُلْكُ وَهُوَ عَلَىٰ كُلِّ شَيْءٍ قَدِيرٌ ﴿١﴾ الَّذِي خَلَقَ الْمَوْتَ وَالْحَيَاةَ لِيَبْلُوَكُمْ أَيُّكُمْ أَحْسَنُ عَمَلًا  وَهُوَ الْعَزِيزُ الْغَفُورُ ﴿٢﴾ [ملک]}
	
	{\ترجمه
		پربرکت و زوال‌ناپذیر است کسی که حکومت جهان هستی به دست اوست، و او بر هر چیز تواناست. (۱) آن کس که مرگ و حیات را آفرید تا شما را بیازماید که کدام یک از شما بهتر عمل می‌کنید، و او شکست‌ناپذیر و بخشنده است. (۲) 
	}
	~\\~
	\item []
	{\قرآن
		
		كَيْفَ تَكْفُرُونَ بِاللَّـهِ وَكُنتُمْ أَمْوَاتًا فَأَحْيَاكُمْ ثُمَّ يُمِيتُكُمْ ثُمَّ يُحْيِيكُمْ ثُمَّ إِلَيْهِ تُرْجَعُونَ [بقره، ۲۸]}
	
	{\ترجمه
		چگونه به خداوند کافر می‌شوید؟ در حالی که شما مردگان بودید، و او شما را زنده کرد. سپس شما را می‌میراند و بار دیگر شما را زنده می‌کند. سپس به سوی او بازگردانده می‌شوید. 
	}
	~\\~
	\item []
	{\قرآن
		قَالُوا رَبَّنَا أَمَتَّنَا اثْنَتَيْنِ وَأَحْيَيْتَنَا اثْنَتَيْنِ فَاعْتَرَفْنَا بِذُنُوبِنَا فَهَلْ إِلَىٰ خُرُوجٍ مِّن سَبِيلٍ [غافر، ۱۱]}
	
	{\ترجمه
		آنها می‌گویند: «پروردگارا! ما را دو بار میراندی و دو بار زنده کردی. اکنون به گناهان خود معترفیم. آیا راهی برای خارج شدن وجود دارد؟» 
	}
	
\end{itemize}

ما معمولاً فراموش می‌کنیم که میلیاردها سال بی‌جان بودیم و اصلاً هم از این مسأله رنجیده‌خاطر نبودیم. زمین شکل گرفت، تمدن‌ها آمدند و رفتند، و ما حتی نمی‌دانستیم چه اتفاقی دارد می‌افتد. و سپس زیستن را تجربه کردیم. تأکید قرآن نیز بر آن است که همان طوری که یک بار قبلاً به دنیا آمدیم، بار دیگر نیز امکان تولد وجود دارد. 

\begin{itemize}
	\item[]
	
	{\قرآن
		أَوَلَمْ يَرَ الْإِنسَانُ أَنَّا خَلَقْنَاهُ مِن نُّطْفَةٍ فَإِذَا هُوَ خَصِيمٌ مُّبِينٌ ﴿٧٧﴾ وَضَرَبَ لَنَا مَثَلًا وَنَسِيَ خَلْقَهُ قَالَ مَن يُحْيِي الْعِظَامَ وَهِيَ رَمِيمٌ ﴿٧٨﴾ قُلْ يُحْيِيهَا الَّذِي أَنشَأَهَا أَوَّلَ مَرَّةٍ وَهُوَ بِكُلِّ خَلْقٍ عَلِيمٌ ﴿٧٩﴾ [یس]
	}
	
	{\ترجمه
		آیا انسان نمی‌داند که ما او را از نطفه‌ای بی‌ارزش آفریدیم؟ و او به مخاصمه آشکار برخاست. (۷۷) و برای ما مثالی زد و آفرینش خود را فراموش کرد و گفت «چه کسی این استخوان‌ها را زنده می‌کند در حالی که پوسیده است؟» (۷۸) بگو «همان کسی آن را زنده می‌کند که نخستین بار آن را آفرید و او به هر مخلوقی داناست.» (۷۹)}
\end{itemize}

در تفسیر شافی\LTRfootnote{\lr{Shafi, Maulana Mufti Muhammad. ``Ma’ariful Qur’an: a Comprehensive Commentary on the Holy Qur’an.’’ 7:413, Translated by Maktaba-e-Darul’Ulum Karachi. 1997--2004.}}
داستانی در مورد آیات فوق آمده است. ما باید به این نکته توجه داشته باشیم که حضرت محمد{\صلی}، اولین معلمِ بشری قرآن، با مخاطبان بدبینی مواجه بود. بدبینی جزئی از طبیعت بشر است و لزوماً مخصوص روزگار ما نیست. در آن داستان آمده است وقتی پیامبر{\صلی} در مورد زندگی پس از مرگ موعظه کرد، یکی از شنوندگان که به این حرف باور نداشت استخوان پوسیده‌ای را برداشت، در دست‌هایش آن استخوان را فشار داد و به گرد تبدیل کرد، و گفت «چه کسی این استخوان‌های پوسیده را زنده می‌کند؟» پاسخ به واضح‌ترین شکل ممکن داده شد: «همان کسی که اولین بار آن استخوان را به وجود آورد، آن را زنده می‌کند.»

یکی از نام‌های خداوند در قرآن  «المصور» است. شکل جسمانی ما هر روز تغییر می‌کند، و این نام خدا به ما خاطرنشان می‌کند که ما اختیاری در مورد شکل جسمانی‌مان نداریم. خودِ من شاهد رشد پسرم  از زمانی که او را از دستگاه سونوگرافی دیدم تا حالا، که روی زمین به راحتی راه می‌رود بودم و به فضل خداوند او  یک روز بزرگ خواهد شد. با وجودی که من و همسرم وسیلهٔ به دنیا آمدن فرزندمان بودیم، در این فرآیند هیچ دخالتی نداشتیم. نوع بیان قرآن در این رابطه روشن است.

\begin{itemize}
	\item[]
	{\قرآن
		ثُمَّ خَلَقْنَا النُّطْفَةَ عَلَقَةً فَخَلَقْنَا الْعَلَقَةَ مُضْغَةً فَخَلَقْنَا الْمُضْغَةَ عِظَامًا فَكَسَوْنَا الْعِظَامَ لَحْمًا ثُمَّ أَنشَأْنَاهُ خَلْقًا آخَرَ  فَتَبَارَكَ اللَّـهُ أَحْسَنُ الْخَالِقِينَ [مؤمنون، ۱۴]}
	
	{\ترجمه
		سپس نطفه را به صورت علقه [خون بسته‌]، و علقه را به صورت مضغه [چیزی شبیه گوشت جویده شده‌]، و مضغه را به صورت استخوان‌هایی درآوردیم و بر استخوان‌ها گوشت پوشاندیم. سپس آن را آفرینش تازه‌ای دادیم. پس بزرگ است خدایی که بهترین آفرینندگان است.
	}
	~\\~
	\item[]
	{\قرآن
		وَاللَّـهُ أَخْرَجَكُم مِّن بُطُونِ أُمَّهَاتِكُمْ لَا تَعْلَمُونَ شَيْئًا وَجَعَلَ لَكُمُ السَّمْعَ وَالْأَبْصَارَ وَالْأَفْئِدَةَ ۙ لَعَلَّكُمْ تَشْكُرُونَ [نحل، ۷۸]}
	
	
	{\ترجمه
		و خداوند شما را از شکم مادرانتان خارج نمود در حالی که هیچ چیز نمی‌دانستید؛ و برای شما، گوش و چشم و عقل قرار داد، تا شکر نعمت او را به جا آورید.  
	}
	
	~\\~
	\item[]
	{\قرآن
		خَلَقَ السَّمَاوَاتِ وَالْأَرْضَ بِالْحَقِّ وَصَوَّرَكُمْ فَأَحْسَنَ صُوَرَكُمْ وَإِلَيْهِ الْمَصِيرُ [تغابن، ۳]}
	
	{\ترجمه
		آسمان‌ها و زمین را به حق آفرید و شما را  تصویر کرد، تصویری زیبا و دلپذیر و سرانجام به سوی اوست. 
	}
	~\\~
	\item[]
	{\قرآن
		وَاللَّـهُ خَلَقَكُمْ ثُمَّ يَتَوَفَّاكُمْ ۚ وَمِنكُم مَّن يُرَدُّ إِلَىٰ أَرْذَلِ الْعُمُرِ لِكَيْ لَا يَعْلَمَ بَعْدَ عِلْمٍ شَيْئًا ۚ إِنَّ اللَّـهَ عَلِيمٌ قَدِيرٌ [نحل، ۷۰]}
	
	{\ترجمه
		خداوند شما را آفرید. سپس شما را می‌میراند. بعضی از شما به نامطلوب‌ترین سنین بالای عمر می‌رسند، تا بعد از علم و آگاهی، چیزی ندانند. خداوند دانا و تواناست. 
	}
\end{itemize}

نکتهٔ مهمی که باید در خاطر داشته باشیم آن است که این  اتفاق یک فرآیند است. همان طور که پیش‌تر اشاره کردم، زمان فقط از دیدگاه انسانی مفهومی مرتبط است. برای خداوند هیچ تجربه‌ای با زمان محدود نمی‌شود. بنابراین حتی اگر به وجود آمدن من یک فرآیند چند میلیارد ساله باشد، باز هم خدا بوده که مرا آفریده است. همین طور اگر روزی پیر شوم و «به بدترین مرحلهٔ زندگی برسم؛ آنچنان که بعد از علم و آگاهی، چیزی ندانم»\footnote
{اشاره به آیهٔ پنجم سورهٔ حج.}  
لزوماً یک فرآیند زوالِ اتفاقی نیست، بلکه فرآیندی است که خداوند آن را هدایت می‌کند.

پیامبر{\صلی}به پیروانش آموخت که هر گاه بیمار شدند به سراغ دارو بروند و در عین حال به خدا توکل کنند. قرن‌های متمادی مسلمانان در مسائل پزشکی پیشرو بودند و امروز نیز بسیاری از پزشکان و محققان داروسازی مسلمان هستند. باور به آفرینندهٔ مرگ و زندگی لزوماً به معنای آن نیست که مؤمنان نباید در مسائلی که در آن‌ها توانایی دخالت دارند، ایفای نقش کنند. اما همان طوری که هر پزشکی می‌داند، موقعیت‌هایی پیش می‌آید که هیچ کاری از دست پزشک ساخته نیست. قرآن در آیات هشتاد و سوم تا هشتاد و هفتم سورهٔ واقعه می‌فرماید:
\begin{itemize}
	\item[]
	{\قرآن
		لَوْلَا إِذَا بَلَغَتِ الْحُلْقُومَ ﴿٨٣﴾ وَأَنتُمْ حِينَئِذٍ تَنظُرُونَ ﴿٨٤﴾ وَنَحْنُ أَقْرَبُ إِلَيْهِ مِنكُمْ وَلَـٰكِن لَّا تُبْصِرُونَ ﴿٨٥﴾ فَلَوْلَا إِن كُنتُمْ غَيْرَ مَدِينِينَ ﴿٨٦﴾ تَرْجِعُونَهَا إِن كُنتُمْ صَادِقِينَ ﴿٨٧﴾}
	
	{\ترجمه
		پس چرا هنگامی که جان به گلوگاه می‌رسد (۸۳) و شما در این حال نظاره می‌کنید (۸۴) و ما از شما به او نزدیک‌تریم ولی نمی‌بینید. (۸۵) اگر هرگز در برابر اعمالتان جزا داده نمی‌شوید (۸۶) پس آن  را بازگردانید اگر راست می‌گویید. (۸۷)}
\end{itemize}


انسان بسیاری از کارهای ظاهراً ناممکن را می‌تواند انجام دهد ولی غلبه بر مرگ کاری نیست که  برای آن راه حلی داشته باشد. همه می‌میرند. مرگ محتوم یک قاعدهٔ ثابت و خلل‌ناپذیر برای همهٔ انسان‌هاست. بنابراین بخش بزرگی از دین ما بر مبنای بینش ما در مورد اتفاقات پس از مرگ است. اگر ما به رستاخیز ایمان نداشته باشیم، کما این که میلیاردها سال قبل مرده بودیم، آن‌گاه بر اساس همین بی‌ایمانی زندگی و عمل می‌کنیم. اگر ما به زندگی رضایت‌بخش و خرم پس از مرگ باور داشته باشیم، آن‌گاه همهٔ تلاش خود را در زندگیِ اکنون صرف آن می‌کنیم که به آن تجربهٔ لذت‌بخش دست یابیم.

بگذارید مثالی بزنم. من عاشق درختان هستم. همیشه در نظر من، درختان زیباترین پدیده در این دنیا هستند. خیلی وقت‌ها هوس می‌کنم به جاهایی بروم که پر از دار و درخت باشد و هیچ کاری غیر از تماشای آن‌ها نداشته باشم. چون برای من تماشای آن‌ها به خودی خود هدفی غایی است. اما یکی از دلایلی که خیلی وقت‌ها از رفتن به جنگل چشم‌پوشی می‌کنم این است که باور دارم زیبایی این درختان در مقایسه با زیبایی درختانی که در آن دنیا شاید ببینم هیچ است. قرآن به درخت سدرةالمنتهی اشاره می‌کند. از نظر علمای اسلامی، این درخت نمایانگر بالاترین جزء خلقت است. به بیانی دیگر، این درخت در جایی از خلقت وجود دارد که به حضور الهی نزدیک‌ترین است - جایی که هیچ حضور جسمانی‌ای معنا ندارد. خوش دارم به لطف خداوند آن درخت را ببینم. بنابراین اگر کوشیدن برای بهبود معنویتم نیازمند حضور در شهر باشد و در نتیجه از دیدن درختان جنگل محروم باشم، اشکالی ندارد؛ چون این زندگی مقدمه‌ای برای چیزی است که «بهتر و پایدارتر»\footnote{اشاره به آیهٔ ۱۷ سورهٔ اعلی.} است.

واقعیت این است که ما هر روز برای مسائل دنیایی عادت به ازخودگذشتگی داریم. شانزده سال به مدرسه می‌رویم که کار مطلوبی پیدا کنیم و خانه‌ای کوچک برای خودمان بخریم. سال‌ها و سال‌ها پس‌انداز می‌کنیم تا فقط به کشور مورد علاقه‌مان مسافرت کنیم. زمان‌هایی را فقط برای گذراندن با کسانی که دوست داریم اختصاص می‌دهیم. پس چرا نباید  در این زندگی برای مرحلهٔ بعدی تلاش کنیم؟

اگر در دینِ کسی باور به زندگیِ پس از مرگ وجود داشته باشد، آن شخص باید کارهای روزانه‌ای را انجام دهد که نمایانگر عملی آن باور است. در غیر این صورت، آن باور صرفاً در مقام حرف باقی می‌ماند. فرض اسلام بر آن است که خداوند بهتر از ما می‌داند که ما  چه کاری را باید انجام دهیم و بهترین راه برای رسیدن به آمرزش ابدی چیست. هر چه باشد، یکی از نام‌های خداوند کریم، یعنی بخشنده، است.  






