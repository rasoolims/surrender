چگونه سخن از دین را آغاز کنیم؟ آیا مثل درس‌های «ادیان جهان»، که در دانشگاه‌ها تدریس می‌شود، عقائد بنیادی و عبادات اصلی ادیان بزرگ را مرور کنیم؟ آیا باید در مورد کاری که دین در جهان می‌کند مانند «خون‌ریزی» یا «تشویق مردم به ازخودگذشتگی» بگوییم؟ راه درست برای آغاز چیست؟ ترجیح من این است که سخنم را با این جمله آغاز کنم که از اصطلاح «رلیجن»\LTRfootnote{Religion}،
که در زبان انگلیسی برای دین استفاده می‌شود، استفاده نخواهم کرد. کلمهٔ «رلیجن» تنها در چند قرن اخیر در زبان انگلیسی استفاده شده است و من نیز مانند پژوهشگر دانشگاه هاروارد، ویلفرد کانت‌ول اسمیث\LTRfootnote{\lr{Smith, Wilfred Cantwell. ``The meaning and End of Religion.’’ Fortress Press: 1991.}}، این واژه را برای ادای معنای دین نارسا می‌دانم. در عوض از همان واژهٔ عربی «دین» که عمری بسیار طولانی‌تر از آن واژهٔ انگلیسی دارد استفاده می‌کنم. واژهٔ «دین» در قرآن، در سورهٔ الکافرون، این گونه بیان شده است:

\begin{itemize}
	\item[]
	
	{\قرآن
		قُلْ يَا أَيُّهَا الْكَافِرُونَ ﴿١﴾ لَا أَعْبُدُ مَا تَعْبُدُونَ ﴿٢﴾ وَلَا أَنتُمْ عَابِدُونَ مَا أَعْبُدُ ﴿٣﴾ وَلَا أَنَا عَابِدٌ مَّا عَبَدتُّمْ ﴿٤﴾ وَلَا أَنتُمْ عَابِدُونَ مَا أَعْبُدُ ﴿٥﴾ لَكُمْ دِينُكُمْ وَلِيَ دِينِ ﴿٦﴾
	}
	
	{\ترجمه
		بگو ای کافران! (۱) آنچه را شما می‌پرستید من نمی‌پرستم. (۲) و نه شما آنچه را من می‌پرستم می‌پرستید (۳) و نه من هرگز آنچه را شما پرستش کرده‌اید می‌پرستم (۴) و نه شما آنچه را که من می‌پرستم پرستش می‌کنید (۵)  آیین شما برای خودتان، و آیین من برای خودم. (۶)
	}
\end{itemize}
‍‍
از نظر واژه‌شناسی، واژهٔ دین مرتبط با «دَین» به معنای بدهی است. همهٔ ما به ارادهٔ‌ دیگران و نه به انتخاب خودمان به دنیا آمده‌ایم. پنداری زندگی می‌کنیم که دِین‌مان را اول از همه نسبت به آن چیزی که موجب به وجود آمدن ماست ادا کنیم. هیچ کدام‌مان نمی‌توانیم بی‌عمل زندگی کنیم زیرا همین که زنده‌ایم به این معنی است که مدام در حال انتخاب کردن و عمل به آن انتخاب‌ها هستیم. هر کسی بر اساس فهم خود در مورد معنای زندگی دست به انتخاب و عمل می‌زند. اگر کسی فکر کند که زندگی بی‌معناست و همهٔ‌ وجود حاصل از یک اتفاق کاملاً تصادفی است، هر کاری که دلش بخواهد می‌کند. اگر کسی همهٔ زندگی‌اش را مدیون والدینش بداند، همهٔ هم و غمش را برای خدمت کردن و شاد کردن والدینش می‌گذارد. اگر من بر این باور باشم که خداوند همان کسی است که مرا به عرصهٔ وجود آورده است، احتمالاً آن گونه زندگی خواهم کرد که به پاس زندگی‌ای که به من عطا شده است، حق‌شناس پروردگار باشم. در همهٔ حالات، انسان همین که به وجود می‌آید دین دارد. بنابراین وقتی که قرآن می‌گوید «آیین شما برای خودتان، و آیین من برای خودم»، در مورد همهٔ انسان‌ها صحبت می‌کند، حتی آن‌هایی که ما آن‌ها را بی‌دین یا به دور از هر گونه معنویت می‌پنداریم.\LTRfootnote{\lr{Chittick, William C. and Sachiko Murata. ``The vision of Islam.’’ pp. xxvii-xxix, Paragon House: 1994.}}

فهم این سخن بسیار اهمیت دارد؛ مخصوصاً در شرایط کنونی که ما دین را یک شأن جدا افتاده از دیگر شئون زندگی می‌بینیم و آن را کاملاً جدا از دیگر جنبه‌های زندگی مانند سیاست، اقتصاد و هنر می‌پنداریم. اسلام هرگز بین امر قدسی و امر عرفی\LTRfootnote{Secular} تفاوتی نمی‌گذارد. در آیهٔ ۱۱۵ سورهٔ بقره آمده است: 

\begin{itemize}
	\item[]
	
	{\قرآن
		وَلِلَّـهِ الْمَشْرِقُ وَالْمَغْرِبُ  فَأَيْنَمَا تُوَلُّوا فَثَمَّ وَجْهُ اللَّـهِ  إِنَّ اللَّـهَ وَاسِعٌ عَلِيمٌ
	}
	
	{\ترجمه
		مشرق و مغرب، از آن خداست. و به هر سو رو کنید، خدا آنجاست. خداوند بی‌نیاز و داناست.
	}
\end{itemize}

همه چیز در زندگی، حتی گفتمان‌های عرفی، امری از حقیقت الهی را بازتاب می‌دهند. فقط این مهم است که درست به آن‌ها بنگریم. در جای دیگری از قرآن، آیهٔ چهارم سورهٔ حدید، آمده است:  {\قرآن وَهُوَ مَعَكُمْ أَيْنَ ما كُنتُمْ  وَاللَّـهُ بِمَا تَعْمَلُونَ بَصِيرٌ}\footnote{و هر جا باشید او با شما است، و خداوند نسبت به آنچه انجام می‌دهید بیناست.}. فرقی نمی‌کند؛ اگر در بازی فوتبالِ آمریکایی باشید، در باشگاه شبانه، در دفتر یک سیاست‌مدار، در اتاق جلسات یک شرکت، در کلیسا یا مسجد، هر جا که باشید، خدا آنجا حضور دارد. خداوند در آیهٔ هفتم سورهٔ‌ مجادله می‌فرماید:

\begin{itemize}
	\item[]
	{\قرآن
		أَلَمْ تَرَ أَنَّ اللَّـهَ يَعْلَمُ مَا فِي السَّمَاوَاتِ وَمَا فِي الْأَرْضِ مَا يَكُونُ مِن نَّجْوَىٰ ثَلَاثَةٍ إِلَّا هُوَ رَابِعُهُمْ وَلَا خَمْسَةٍ إِلَّا هُوَ سَادِسُهُمْ وَلَا أَدْنَىٰ مِن ذَٰلِكَ وَلَا أَكْثَرَ إِلَّا هُوَ مَعَهُمْ أَيْنَ ما كَانُوا ۖ ثُمَّ يُنَبِّئُهُم بِمَا عَمِلُوا يَوْمَ الْقِيَامَةِ إِنَّ اللَّـهَ بِكُلِّ شَيْءٍ عَلِيمٌ
	}
	
	{\ترجمه
		آیا نمی‌دانی که خداوند آنچه را در آسمان‌ها و آنچه را در زمین است می‌داند؛ هیچ گاه سه نفر با هم نجوا نمی‌کنند مگر اینکه خداوند چهارمین آن‌هاست، و هیچ‌گاه پنج نفر با هم نجوا نمی‌کنند مگر اینکه خداوند ششمین آن‌هاست، و نه تعدادی کمتر و نه بیشتر از آن مگر اینکه او همراه آن‌هاست هر جا که باشند، سپس روز قیامت آنها را از اعمالشان آگاه می‌سازد، چرا که خداوند به هر چیزی داناست. 
	}
\end{itemize}

البته قاعدتاً این حرف به این معنا نیست که ما همیشه در احاطهٔ شعائر مذهبی مانند اذان، که هر روز چندین بار در کشورهای اسلامی خوانده می‌شود، هستیم. بلکه ما با جلوه‌های بی‌نهایت توانایی خلاقانهٔ پروردگار مواجهیم. وقتی به یک نمایشگاه هنری می‌رویم،‌ از مهارت‌هایی که یک هنرمند با آن‌ها اثر هنری‌اش را به وجود آورده شگفت‌زده می‌شویم. اگر همان هنرمند در آن نمایشگاه حضور داشته باشد، شاید از او دربارهٔ مبدأ این خلاقیت و روش‌هایی که این اثر هنری را ساخته است بپرسیم. ممکن است هیجان‌زده شویم و به هنرمند بگوییم «عاشق کارش هستیم.»  دلیل این هیجان ما این است که آن هنرمند را مبدأ آن هنر می‌دانیم. چیزی منحصر به فرد در مورد خلاقیت هنرمندان وجود دارد که می‌خواهیم آن را بفهمیم و بستاییم. خداوند بسیار بالاتر از این‌هاست. همان گونه که به زبان عربی می‌گوییم  «الله اکبر»، خداوند از یک هنرمند بزرگ‌تر است، زیرا که خلاقیت یک هنرمند تنها قطره‌ای از اقیانوس آفرینش الهی است. خداوند از هر دانشمندی بزرگ‌تر است، زیرا خِرد یک دانشمند قطره‌ای از خِرد مطلق است. خداوند از هر زن یا مرد زیبایی برتر است، زیرا آن زیباییِ ظاهری آفریدهٔ زیباییِ پروردگار است، پروردگاری که نه زاده و نه زاییده شده است. 

دین را نباید هر چیزی غیر از کلیتِ وجود بدانیم. اگر تخمین کنونی ما از عمر کائنات، که ۱۳٫۸ میلیارد سال است، درست باشد، حتماً خداوند در همهٔ زمان‌ها و فضاها بوده است. یکی از معلم‌هایم یک بار به من گفت: «برای خداوند هیچ لحظه‌ای از چهارده میلیارد سال کمتر نیست و هیچ اتمی از کهکشان کوچک‌تر نیست، زیرا خداوند فرای زمان و مکان وجود دارد و خود او خالق زمان و مکان است.» تأمل در دین نه تنها تجربه‌ٔ ما را در مورد وجود محدود نمی‌کند بلکه به ما این فرصت را می‌دهد که همهٔ‌ چیزهایی را که بود و هست و خواهد بود دریابیم. در عین حال، دین همیشه یک واقعیت ناگزیر را به ما گوش‌زد می‌کند: زمانِ ما بر روی زمین تمام‌شدنی است. یک روز باید دِین خود را ادا کنیم. دین آمده تا ما را برای آن روز آماده کند. 







