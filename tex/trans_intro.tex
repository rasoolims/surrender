 {\مقدمه
	اگر نگوییم که در آمریکا، مخصوصاً در فضای دانشگاهی و روشنفکری، «خدا مرده است»، لااقل می‌توانیم ادعا کنیم که تفکر در مورد امر قدسی مسأله‌ای حاشیه‌ای است. البته در این میانه افرادی پیدا می‌شوند که شنا کردن برخلاف جریان آب را خوب بلدند. در میان تازه‌مسلمانانی که دیده‌ام، شاید شاخص‌ترین‌شان نویسندهٔ همین کتاب باشد.  قرن‌ها پیش، اجدادش از اسکاتلند به آمریکا مهاجرت کرده‌اند و به قول خودش، در محله‌ای زندگی کرده که حتی یک غیرسفید‌پوست وجود نداشته است. در نوجوانی به صورت اتفاقی با قرآن آشنا و مسلمان می‌شود و بعد از حضور در جلسات محرم در شهر نیویورک رابطهٔ عمیق احساسی با اهل بیت {\علیهم} پیدا می‌کند. من دورادور تنها شاهد تغییر رفتارهای او در سال‌های اخیر، مخصوصاً پیش و پس از سفر زیارتی‌اش به نجف و کربلا، بوده‌ام:

\textit{
	«هر کسی، چه سنی و چه شیعه، به این مسأله اذعان دارد که هیچ کس به بزرگی امام علی{\علیه} نیست؛ مقام حضرت محمد{\صلی} که جای خود دارد [و بالاتر از همهٔ انسان‌هاست]. وقتی موقع زیارت امام علی{\علیه} دو رکعت نماز خواندم، اشک بی‌اختیار از چشمانم می‌آمد.  تمام آنچه از امام علی{\علیه} می‌دانستم چون سیل به سوی قلبم روانه شده بود. همهٔ بلاها، مصیبت‌ها و دشواری‌هایی که ایشان متحمل شده بود. وفاداری، پایمردی و عزمش در ادامه دادن کاری که باید به سرانجام می‌رسید بی آن که به سختی آن کار فکر کند. آن وقت بود که فهمیدم و اکنون به یاد می‌آورم، اگر همهٔ لحظه‌هایم را در مسیر حقیقت استفاده کنم، به اندازهٔ یک قطره از اقیانوس ابوتراب{\علیه} نخواهد شد.»}\LTRfootnote{
		\lr{{\tiny \url{amercycase.com/2018/08/12/leaving-on-a-jet-plane/}}}}
	
	وقتی پای صحبتش می‌نشستم به ایمانش غبطه می‌خوردم؛ پنداری خداوند او را به دنیا آورده که یادمان بیاورد حتی در اوج غفلت و فروبستگی امر قدسی، می‌شود دیندار ماند و هیچ بهانه‌ای برای غفلت پذیرفته نیست. همین نوشتهٔ بالا که از وبلاگش برداشته‌ام، نامهٔ خداحافظی‌اش از نیویورک برای شروع تحصیل دکتری برکلی است با این توضیح که قبل از شروع سال تحصیلی به بنگلادش می‌رود تا برای کمک به مسلمانان آوارهٔ روهینگیا آستین همت  بالا بزند.
	
	بلافاصله پس از انتشار کتاب و خواندن آن، از او خواستم که به من اجازهٔ ترجمهٔ کتاب را بدهد. دست به کار شدم و سعی کردم هر آن چه بلدم روی داریهٔ ترجمه بریزم. متأسفانه  در ترجمه تخصصی ندارم و با این کمیتِ لنگ، خوب می‌دانم که آب در هاون کوبیده‌ام. امیدوارم این توضیحاتْ عذر بدتر از گناهِ ترجمهٔ ضعیف نباشد. عنوان اصلی کتاب به شکل زیر است که ترجیح دادم به ترجمهٔ تحت‌اللفظی آن وفادار نمانم:
	\begin{center}
{\small		\lr{``The Beautiful Surrender: Islam as a Path to be Walked’’} }
	\end{center}
	
	در فرآیند ترجمه سعی کردم در بسیاری از جاها، با سعی در ضربه نزدن به پیام اصلی، دست به ترجمهٔ آزاد بزنم. ترجمه‌های قرآنی همه از سایت تنزیل و از روی ترجمهٔ آیت‌الله مکارم شیرازی، پس از برداشتن توضیحات داخل کمانک، اخذ شده است. متن عربی قرآن نیز از روی سایت تنزیل برداشته شده است. متن این کتاب را به قالب XeTex در وب‌سایت گیت‌هاب‌‌ به صورت متن‌باز گذاشته‌ام به آن امید که اگر خواهان دسترسی به متن اصلی برای تبدیل به قالب‌هایی غیر از پی‌دی‌اف یا تغییر نوع و اندازهٔ قلم باشید، با دردسر کمتری مواجه شوید. علاوه بر این، امکان یادآوری ایرادهای ترجمه از طریق سامانهٔ پیگیری خطاها در این وبگاه وجود دارد:
	\begin{center}
		\lr{\tiny{\url{github.com/rasoolims/surrender}}}
	\end{center}
	در نهایت جا دارد از برادر بزرگوار آقای محمدحسین باطنی برای مطالعهٔ پیش‌نویس این ترجمه سپاسگزاری کنم. 
	
	\begin{center}
		هر چه گفتیم جز حکایت دوست
		
		در همه عمر از آن پشیمانیم
	\end{center}
	\begin{flushleft}
		{\ترجمه
			محمدصادق رسولی
			
			تابستان ۱۳۹۷ ه. ش.،	نیویورک
		}
	\end{flushleft}
	
}











