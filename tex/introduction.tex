تابستان ۱۹۹۷، پس از اتمام دبیرستان شبانه‌روزی‌ و پیش از ورود به دانشگاه، شروع به خواندن قرآن کردم. قبلاً در مدرسهٔ شبانه‌روزی بخش‌های منتخبی از قرآن را در کلاس «ادیان غربی» خوانده بودم اما این مرتبه فرق می‌کرد. من یک نسخه از قرآنی را در دست داشتم که برایم قابل فهم بود، می‌توانستم حتی تصادفی بخشی از آن را باز کنم و بی هیچ پیش‌نیازی بخوانم و در مورد آنچه خوانده‌ام تأمل کنم. قرآنی که داشتم نسخهٔ ویرایش‌شدهٔ ترجمهٔ مارمادوک پیکثال\LTRfootnote{\lr{Marmaduke Pickthall}}
بود که اولین بار در سال ۱۹۳۰ میلادی در لندن منتشر شده بود. روزی که دو نفر از آشناهای مسلمان‌شده‌ام را در یکی از اجراهای محلی موسیقیِ پانک‌ راک\LTRfootnote{\lr{Punk rock}}
دیدم، آن‌ها این کتاب را به من توصیه کردند و من از کتاب‌فروشی اسلامی خیابان دِوُن\LTRfootnote{\lr{Devon}} 
شهر شیکاگو این کتاب را خریدم.


قرآن شبیه هیچ کدام از کتاب‌هایی که قبلاً خوانده بودم نبود. هر چه بیشتر می‌خواندم، بیش از پیش درمی‌یافتم که این کتاب را یک انسان ننوشته است. پیامی عمیق در متن قرآن بود که مرا به تأمل وامی‌داشت. باعث می‌شد همیشه خود را نسبت به قرآن حقیر ببینم. مکرراً قرآن را می‌خواندم و سعی می‌کردم آن را و  احساسات خود را نسبت به آن بفهمم. من اسیر تجلی  خِرد برتر موجود در قرآن شده بودم. 

بعد از سال اول کارشناسی، شغلی برای تابستان دست و پا کردم و به عنوان مشاور پسران نوجوان چهارده تا شانزده ساله استخدام شدم. موقعی که به اردوگاه رفتم، قرآنم را همراه خود بردم. در همان تابستان، من و یکی دیگر از همکارانم دو سفر با کانو ترتیب دادیم؛‌ اولی‌اش یک سفر نه‌روزه به سمت نیویورک و دیگری یک سفر دو هفته‌ای به سمت منطقهٔ حیات وحش کِبِک. موقع سفر دوم بود که وارد مرحلهٔ اسلام آوردن شدم. هر روز بعد از راست و ریست کردن کارهای اردوگاه، ده دقیقه‌ای کنجی خلوت می‌جستم تا در آیات قرآن درنگ کنم. کنار رودخانه‌ها و برکه‌ها می‌نشستم، به درختان و آسمان چشم می‌دوختم و قرآن را باز می‌کردم. هر روز یک فکر به ذهنم خطور می‌کرد: اگر معنایی ذاتی در زیبایی کائنات باشد، درست همان معنایی است که قرآن می‌خواهد به ما بگوید. وقتی در کنار آن رودخانه‌ها و برکه‌ها قرآن می‌خواندم و نظاره‌گر پدیدار شدن ستاره‌ها در آسمان تیره می‌شدم، در وجودم حسی نو پدیدار می‌شد و آن حسْ چیزی جز ایمان نبود.


می‌توانم به بخش‌هایی از قرآن ارجاع بدهم تا منظورم را برسانم. اما این کتاب تلاش یک خردِ ناکامل برای بیان تجربه‌اش در مورد خِرَدی است که همهٔ موجودات شناخته و ناشناخته را به عرصهٔ‌ وجود آورده است. این کتاب نه دربارهٔ من، بلکه دربارهٔ اسلام است. این نوشته تلاش حقیرانه‌ای است برای طلب رحمت از کسی که مرا آفریده است؛ همان کسی که آن ترجمهٔ‌ قرآن را در تابستان ۱۹۹۷ به دستم رساند. 

\enlargethispage{2\baselineskip}
  \begin{samepage}
\begin{flushleft}
	{\ترجمه
	
	ر. دیوید کولیج
	
	رمضان ۱۴۳۹ هجری قمری،	ژوئن ۲۰۱۸ میلادی
	
	نیویورک
}
\end{flushleft}
\end{samepage}






