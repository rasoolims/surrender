تقوا واژه‌ای عربی است که معنای اصلی‌اش بیش از دویست بار در قرآن آمده و به معنای تغییر رفتار به خاطر آگاهی از حضور خداوند است. تمایل به تغییر ممکن است به دلیل عشق، امید، ترس، سپاس‌گزاری، و انواع مختلف دلایل احساسی باشد. اما به خاطر آن که «ترس از خداوند» در گذشته زیاد استفاده شده است، معمولاً تقوا به عنوان ترس از خداوند تعبیر می‌شود. من از واژهٔ اصلی عربی استفاده می‌کنم، چون مانند دین، تقوا یک مفهوم بنیادین در اسلام است و معناهای مختلفی را در آن واحد در بر می‌گیرد.

یکی از راه‌های فهمیدن تقوا تقسیم آن به پنج گام است.\footnote{این مفهوم را ابن جوزی کلبی در تفسیر قرآنش استفاده کرده است. این تقسیم‌بندی در این کتاب نیز آمده است:
	
	\lr{Shakir, Zaid and Hamza Yusuf. ``Agenda to Change our Condition.’’ pp 12-13, Sandala, 2013.}}
گام اول تسلیم یا سرسپردگی (اسلام) است. در فصل قبل، در مورد اسلام به عنوان یک پدیدهٔ جهان‌شمول صحبت کردیم. در اینجا مرادمان از اسلامْ پذیرش نبوت حضرت محمد{\صلی}به عنوان آورندهٔ قرآن برای بشریت و معلم نماز، کار خیر، روزه و زیارت حج است. مثلاً پیروی دین حضرت ابراهیم{\علیه}تقریباً ناممکن است. زیرا ما چیز زیادی از زندگی ایشان، نحوهٔ عبادت و زمان روزه گرفتن‌شان نمی‌دانیم. هر سال، میلیون‌ها مسلمان بر اساس تعالیم پیامبر اسلام{\صلی}در ماه رمضان روزه می‌گیرند، و چند میلیون از مسلمانان به سفر حج می‌روند. همین مثال‌ها نشان می‌دهد که پیروی از آیینِ پیامبر اسلام{\صلی}امری دست‌یافتنی است. خودِ من به عنوان یک آمریکایی که تا شانزده سالگی چیزی از نام محمد{\صلی}نشنیده بودم، شاهد زنده‌ای برای این استدلال هستم.

هر وقت کسی راه حضرت محمد{\صلی}را پذیرفت، می‌تواند قدم به دومین گام تقوا، یعنی توبه، بنهد. توبه به معنای رها کردن هر چیزی است که خداوند ما را از آن منع کرده است و انجام دادن هر کاری است که بر ما واجب نموده است. مثلاً، همان طور که در فصل پیش گفتیم، خداوند بر ما واجب کرده که روزی پنج مرتبه نماز بخوانیم. به جز حالات استثنایی مانند زمان عادت ماهانهٔ زنان، همهٔ مسلمانان باید روزی پنج مرتبه نماز بخوانند. ترک نماز از روی قصد گناه محسوب می‌شود و یک مسلمان همیشه با نیروهای درونی و بیرونی که خواندن نماز را برای او دشوار می‌کند در کشمکش است. حال اگر مسلمانی نماز نخواند چطور؟ پاسخ آن است که باید توبه کند. توبه فقط به معنای طلب آمرزش از خداوند نیست، بلکه شامل تغییر نیز می‌شود. مثلاً در مورد نماز خواندن، توبهٔ کسی پذیرفته نمی‌شود مگر این که به عادت نماز خواندن بازگردد. اکتفا کردن به طلب آمرزش از پروردگار برای توبه کافی نیست. به همین صورت، اگر کسی مشغول به کار حرام است، باید آن کار را ترک کند. مثلاً من عادت به نوشیدن الکل داشتم و از نوشیدنش لذت می‌بردم. دلیل من برای ترک الکل ضرر جانی یا عواقب آن برای زندگی‌ام نبود. من نوشیدن الکل را ترک کردم چون بر این باور بودم که خداوند ما را از نوشیدن آن منع کرده است. هر گناهی، چه گناهی که باعث اذیت دیگران شود چه گناهی که صرفاً امری شخصی باشد، مستوجب توبه است.

گام سوم تقوا «ورع» است. ورع به این معناست که باید در مورد زیست دینی خود محتاط باشیم. در مورد گام دوم، گناه بودن کارهایی مانند نوشیدن الکل واضح است اما چیزی که در آن مسأله اهمیت دارد انجام دادن یا ندادن آن کار است. در ورع، فراتر از گناه‌های ظاهری، باید در قبال  کارهای مباح زندگی‌مان محتاطانه رفتار کنیم. فرض کنید کسی به شما یک میلیون دلار بدهد و ادعا کند که از آن پول، قطعاً ده هزار دلارش پول دزدی است و احتمال دارد که ده هزار دلار دیگرش نیز پول دزدی باشد و پلیس فردا برای گرفتن پول دزدیده‌شده می‌آید. واکنش شما چه خواهد بود؟ وقتی پلیس بیاید، آیا شما محض محکم‌کاری بیست هزار دلار می‌دهید یا شروع می‌کنید به بحث کردن که تنها ده هزار دلار از آن پول دزدیده شده و آن‌ها حقی در قبال ده هزار دلارِ دیگر ندارند؟ شرط احتیاط آن است که شما بیست هزار دلار بدهید. به همین شکل، اگر ما در مورد ارتباطمان با خداوند جدی باشیم، همهٔ سعی‌مان را می‌کنیم تا رضایت خداوند در همهٔ کارهایمان در نظر گرفته شود، حتی اگر آن کارها جزو کارهای مباح باشد. اگر شما کسی را دوست داشته باشید و احتمال بدهید گفتن بعضی از حرف‌ها آن شخص را می‌رنجاند، احتمالاً از گفتن آن حرف‌ها اجتناب خواهید کرد، حتی اگر احتمالش کم باشد. اصلاً چرا باید خطر کنیم و کسی را که دوست داریم برنجانیم؟ در ارتباط با پروردگار، این گام از تصفیهٔ اخلاقی و معنوی برای خیلی از مردم قابل دسترسی نیست، اما برای کسانی که برای  رابطه‌شان با خداوند حداقل به اندازهٔ رابطه‌شان با کسانی که دوست‌شان دارند، اهمیت قائل‌اند جایگاه بسیار ویژه‌ای دارد.

گام چهارم تقوا، دل کندن از دنیا یا «زهد» است. اگر کسی را واقعاً دوست داشته باشید، فقط دوست دارید که با او باشید. دوست دارید کم‌تر بخوابید که بیشتر با او باشید، کم‌تر بخورید که بیشتر با او باشید، با دیگران کم‌تر حرف بزنید که بیشتر مشغول صحبت با او باشید. هیچ وقت شما از خواب، خوراک و صحبت کردن‌ با دیگران نمی‌گذرید مگر آن که خودتان خواسته باشید؛ چون به نظرتان مسألهٔ مهم‌تری برایتان وجود دارد. دلیلِ دل کندن از دنیا عشق به خداوند است. کسی که زهد می‌ورزد، بیشتر غرق خواندن قرآن می‌شود تا آن که تاریخ یا رمان بخواند. بیشتر وقت آخر شبش را برای عبادت می‌گذراند و از راحتی رخت‌خواب می‌گذرد. به جای آن که به فکر درست کردن غذاهای لذیذ باشد، بیشتر از حد معمول روزه می‌گیرد. اگر تمکن مالی داشته باشد، بیشتر از آن که برای خودش خرج کند، برای کار خیر خرج می‌کند. اما همهٔ این‌ها به خاطر عشق و آرزوی کسی است که مراد همهٔ جویندگان است.

اگر کسی از چهار گام قبلی بگذرد، به گام پنجم تقوا می‌رسد که همانا دیدن خداوند یا «شهادت» است. به خاطر آن که فرسنگ‌ها با این سطح از تقوا فاصله دارم، صادقانه می‌گویم که هیچ نظری در مورد آن ندارم. 

این پنج گام نشان‌دهندهٔ یک چیز است: اسلام یک مسیر است. سفر بدون گام به گام پیش رفتن امکان‌پذیر نیست. لذا در این مسیر، مهم‌ترین مسأله این است که  بدانیم گام بعدی چیست. اگر نیاز به استراحت دارید،‌ استراحت می‌کنید، و سپس بلند می‌شوید و به تلاش ادامه می‌دهید. اگر من چنین بینشی نداشتم، هیچ وقت عربی را نمی‌آموختم، در خواندن قرآن تسلط پیدا نمی‌کردم و خواندن نماز برایم ساده نمی‌شد. همهٔ این‌ها برای من زمان برد که البته صرف زمان برای این مراحل مسأله‌ای طبیعی است. یکی از معلمانم یک روز به من گفت «مهم آن نیست که چقدر از مسیر پیش رویت باقی مانده، مهم آن است که چقدر از مسیر را تا حالا رفته‌ای.»









