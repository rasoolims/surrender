روزی دانشمندی بزرگ پیش استاد معنوی‌اش رفت تا از او در مورد مسیر معنوی‌اش نصیحتی بشنود. استاد گفت که انسان همیشه در یکی از چهار حالت است و پاسخی درخور برای هر حالت وجود دارد:

\begin{itemize}
	\item[]
	«اگر خوشبخت است، باید شاکر باشد. اگر درگیر مشکلات است، صبر پیشه کند. اگر گناهکار است، توبه کند. اگر مطیع است، تواضع داشته باشد.»
\end{itemize}



کوتاهِ سخن این که هر لحظه‌ از زندگی مهم است، فارغ از آن که چه کاری انجام می‌دهیم. خدا آن طرف اقیانوس نیست؛ در همین‌جاست، در همین نزدیکی است، در همین لحظه، و منتظر ماست که به راه راست برویم.

من نمی‌توانم بگویم شما در کدام یک از چهار حالت مذکور هستید. حتی نمی‌توانم شما را مجبور به واکنش مناسب در هر کدام از آن چهار حالت کنم.  من فقط تلاش کردم که نگرش اسلامی در مورد وجود انسان را به تصویر بکشم. حالا این شمایید که باید معنای زندگی‌تان را دریابید. به همین خاطر است که  در عنوان این کتاب از واژهٔ مسیر استفاده کرده‌ام. بنای این کتاب بر ساختن یک مفهوم انتزاعی نیست، بلکه روشی است که از روی آن بفهمید کجا هستید، معنای مسیری که می‌روید چیست، و راه‌حل چه باید باشد. مثلاً می‌توانستم یک فصل را اختصاص به دین‌شناسی اسلامی بدهم. اما بیشتر مردم به این موضوعات علاقه‌ای ندارند و حتی نیازی هم به آن مباحث ندارند. آنچه که من بر آن تمرکز کردم، یعنی دین، مرگ، شعائر اصلی اسلامی و تقوا، برای همهٔ انسان‌ها اهمیت دارد.

آن چهار حالت به یادمان می‌آورد که ما همیشه در درون‌مان در حال دگرگونی هستیم. یک مسلمان مطیع ممکن است در دام غرور بیفتد حال آن که کسی که هیچ چیز در مورد اسلام نمی‌داند ناگهان تبدیل به خادم بزرگی برای خداوند شود. مراحل تقوا تدریجی است، اما آن چهار حالت تجربه‌ای است که روزاروز بر ما اتفاق می‌افتد. ممکن است صبح نیازمند شکرگزاری باشیم و عصر نیازمند صبر. بنابراین سفر معنوی ما فراتر از داشتن ظاهر مذهبی  است. البته داشتن ظاهر مذهبی نیز به خودی خود اهمیت به‌سزایی دارد. سعی من در این کتاب آن بوده است که نقش اساسی تعلیمات ظاهری اسلام را در رشد معنوی انسان ترسیم نمایم. البته گنجینهٔ دانش عظیمی در مورد زندگیِ باطنی در سنت اسلامی وجود دارد که آن چهار حالتی که گفتم تنها گوشهٔ بسیار ناچیزی از آن گنجینهٔ بزرگ است.

اگر چیزی در این کتاب وجود دارد که به شما می‌قبولاند امکان بهتر شدن وجود دارد، از من تشکر نکنید؛ سپاس‌گزار خداوند باشید. خداوند حقیقت مطلق، الحق، است و حقیقت از سمت خداوند می‌آید. اگر چیزی در این کتاب وجود دارد که منافی حقیقت است، همه‌اش به خاطر نقص من به عنوان یک انسان جایزالخطاست. برای‌تان و برای خودم دعا می‌کنم که بتوانیم سرسپردهٔ او باشیم؛ اویی که  «شما را از شکم مادرانتان خارج نمود در حالی که هیچ چیز نمی‌دانستید. و برای شما، گوش و چشم و عقل قرار داد، تا شکر نعمت او را به جا آورید.»\footnote{سورهٔ نحل، آیهٔ ۷۸.}

اسلام مانند بقیهٔ سنت‌های دینیِ بزرگ دارای مفاهیم غریبه و پیچیده نیز است. فرهنگ مسلمانان جهان نیز مختلف است و به اندازهٔ عمر چندین انسان طول می‌کشد تا آن فرهنگ‌ها را بشناسیم. سعی من در این کتابِ کوتاه نشان دادن همهٔ اسلام یا آنچه مسلمانان باور دارند نبوده، بلکه نشان دادن فرصتی به شما برای انتخاب مسیر درست برای زندگی بوده است، همان طور که هجده سال پیش این فرصت برای من پیش آمد.

\begin{itemize}
	\item[]
	{\قرآن
		وَنَزَعْنَا مَا فِي صُدُورِهِم مِّنْ غِلٍّ تَجْرِي مِن تَحْتِهِمُ الْأَنْهَارُ  وَقَالُوا الْحَمْدُ لِلَّـهِ الَّذِي هَدَانَا لِهَـٰذَا وَمَا كُنَّا لِنَهْتَدِيَ لَوْلَا أَنْ هَدَانَا اللَّـهُ  لَقَدْ جَاءَتْ رُسُلُ رَبِّنَا بِالْحَقِّ  وَنُودُوا أَن تِلْكُمُ {\bfseries الْجَنَّةُ أُورِثْتُمُوهَا بِمَا كُنتُمْ تَعْمَلُونَ} [اعراف، ۴۳]} 
	
	{\ترجمه
		و آنچه در دل‌ها از کینه و حسد دارند، برمی‌کنیم و از زیر آن‌ها، نهرها جریان دارد. می‌گویند «ستایش مخصوص خداوندی است که ما را به این رهنمون شد. و اگر خدا ما را هدایت نکرده بود، ما راه نمی‌یافتیم. مسلّماً فرستادگان پروردگار ما حق را آوردند» و به آنان ندا داده می‌شود که: «این بهشت را در برابر اعمالی که انجام می‌دادید، به ارث بردید.» }
\end{itemize}







